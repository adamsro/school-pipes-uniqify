% Robert Adams	02/07/2012	CS 311

\documentclass[letterpaper,10pt]{article} %twocolumn titlepage 
\usepackage{graphicx}
\usepackage{amssymb}
\usepackage{amsmath}
\usepackage{amsthm}

\usepackage{alltt}
\usepackage{float}
\usepackage{color}
\usepackage{url}

\usepackage{balance}
\usepackage[TABBOTCAP, tight]{subfigure}
\usepackage{enumitem}
\usepackage{pstricks, pst-node}


\usepackage{geometry}
\geometry{margin=1in, textheight=8.5in} %textwidth=6in

%random comment

\newcommand{\cred}[1]{{\color{red}#1}}
\newcommand{\cblue}[1]{{\color{blue}#1}}

\usepackage{hyperref}

\def\name{Robert M Adams}

%% The following metadata will show up in the PDF properties
\hypersetup{
  colorlinks = true,
  urlcolor = black,
  pdfauthor = {\name},
  pdfkeywords = {cs311 ``operating systems'' files filesystem I/O},
  pdftitle = {CS 311 Project 2: Sorting Pipes},
  pdfsubject = {CS 311 Project 1},
  pdfpagemode = UseNone
}

\begin{document}
  \title{CS311 Project 2: Sorting Pipes}
  \author{Robert M. Adams}
  \maketitle


\section{Design Decisions}


    To try to write code which was possibly, slightly, more modular I wrote all the primary functionality into a class. Aside from looking cleaner, I am now able to throw Exceptions and handle the formatting all in the same place. The process really rather procedural by nature though, so the run method really only wraps a series of functions which must execute in order.


    I used a struct to hold the child’s id and all pipes then stored a copy of this struct in a vector for every pipe created. This made it much easier for me to keep things organized.  I also created a typedef to group the read and write file descriptors for a process, which I felt made things a bit more clear. I ended up, however, using malloc for the file descriptor array because I wasn’t able to get this working with a vector... probably because I couldn't figure out the syntax.


    As I pulled data from the sort processes pipes I stored the resulting words in a vector. I then iterated through the vector with std::sort, which with the GNU compiler is apparently introsort, about n log n complexity. I then used std::vector<T>.erase with std::unique to remove all duplicates, probably a bit more then ‘n’ time. All and all the process of calling sort multiple times, then loading into an array and calling sort again seems very inefficient, I wasn’t, however, able to come up with a faster way.

\section{Difficulties}


    Keeping track of which pipes go where and when to close what was definitely confusing. If I were to do this again I would try fifos. On a related matter I still haven't figured out why some of my processes are not terminating.  My loop eventuality calls break; and unleashes some zombies.


    When timing the results I found that I can call up to 60 sort processes on smaller files, and the timings are about the same as 1 sort process. I cannot call any more then one sort process on larger files. Although only speculation, I believe this have something  to do with calling sort on the vector. 

  \begin{figure}[p]
    \centering
    % GNUPLOT: LaTeX picture
\setlength{\unitlength}{0.240900pt}
\ifx\plotpoint\undefined\newsavebox{\plotpoint}\fi
\sbox{\plotpoint}{\rule[-0.200pt]{0.400pt}{0.400pt}}%
\begin{picture}(1500,900)(0,0)
\sbox{\plotpoint}{\rule[-0.200pt]{0.400pt}{0.400pt}}%
\put(130.0,131.0){\rule[-0.200pt]{4.818pt}{0.400pt}}
\put(110,131){\makebox(0,0)[r]{ 0}}
\put(1419.0,131.0){\rule[-0.200pt]{4.818pt}{0.400pt}}
\put(130.0,222.0){\rule[-0.200pt]{4.818pt}{0.400pt}}
\put(110,222){\makebox(0,0)[r]{ 0.5}}
\put(1419.0,222.0){\rule[-0.200pt]{4.818pt}{0.400pt}}
\put(130.0,313.0){\rule[-0.200pt]{4.818pt}{0.400pt}}
\put(110,313){\makebox(0,0)[r]{ 1}}
\put(1419.0,313.0){\rule[-0.200pt]{4.818pt}{0.400pt}}
\put(130.0,404.0){\rule[-0.200pt]{4.818pt}{0.400pt}}
\put(110,404){\makebox(0,0)[r]{ 1.5}}
\put(1419.0,404.0){\rule[-0.200pt]{4.818pt}{0.400pt}}
\put(130.0,495.0){\rule[-0.200pt]{4.818pt}{0.400pt}}
\put(110,495){\makebox(0,0)[r]{ 2}}
\put(1419.0,495.0){\rule[-0.200pt]{4.818pt}{0.400pt}}
\put(130.0,586.0){\rule[-0.200pt]{4.818pt}{0.400pt}}
\put(110,586){\makebox(0,0)[r]{ 2.5}}
\put(1419.0,586.0){\rule[-0.200pt]{4.818pt}{0.400pt}}
\put(130.0,677.0){\rule[-0.200pt]{4.818pt}{0.400pt}}
\put(110,677){\makebox(0,0)[r]{ 3}}
\put(1419.0,677.0){\rule[-0.200pt]{4.818pt}{0.400pt}}
\put(130.0,768.0){\rule[-0.200pt]{4.818pt}{0.400pt}}
\put(110,768){\makebox(0,0)[r]{ 3.5}}
\put(1419.0,768.0){\rule[-0.200pt]{4.818pt}{0.400pt}}
\put(130.0,859.0){\rule[-0.200pt]{4.818pt}{0.400pt}}
\put(110,859){\makebox(0,0)[r]{ 4}}
\put(1419.0,859.0){\rule[-0.200pt]{4.818pt}{0.400pt}}
\put(130.0,131.0){\rule[-0.200pt]{0.400pt}{4.818pt}}
\put(130,90){\makebox(0,0){ 1}}
\put(130.0,839.0){\rule[-0.200pt]{0.400pt}{4.818pt}}
\put(186.0,131.0){\rule[-0.200pt]{0.400pt}{2.409pt}}
\put(186.0,849.0){\rule[-0.200pt]{0.400pt}{2.409pt}}
\put(261.0,131.0){\rule[-0.200pt]{0.400pt}{2.409pt}}
\put(261.0,849.0){\rule[-0.200pt]{0.400pt}{2.409pt}}
\put(299.0,131.0){\rule[-0.200pt]{0.400pt}{2.409pt}}
\put(299.0,849.0){\rule[-0.200pt]{0.400pt}{2.409pt}}
\put(317.0,131.0){\rule[-0.200pt]{0.400pt}{4.818pt}}
\put(317,90){\makebox(0,0){ 10}}
\put(317.0,839.0){\rule[-0.200pt]{0.400pt}{4.818pt}}
\put(373.0,131.0){\rule[-0.200pt]{0.400pt}{2.409pt}}
\put(373.0,849.0){\rule[-0.200pt]{0.400pt}{2.409pt}}
\put(448.0,131.0){\rule[-0.200pt]{0.400pt}{2.409pt}}
\put(448.0,849.0){\rule[-0.200pt]{0.400pt}{2.409pt}}
\put(486.0,131.0){\rule[-0.200pt]{0.400pt}{2.409pt}}
\put(486.0,849.0){\rule[-0.200pt]{0.400pt}{2.409pt}}
\put(504.0,131.0){\rule[-0.200pt]{0.400pt}{4.818pt}}
\put(504,90){\makebox(0,0){ 100}}
\put(504.0,839.0){\rule[-0.200pt]{0.400pt}{4.818pt}}
\put(560.0,131.0){\rule[-0.200pt]{0.400pt}{2.409pt}}
\put(560.0,849.0){\rule[-0.200pt]{0.400pt}{2.409pt}}
\put(635.0,131.0){\rule[-0.200pt]{0.400pt}{2.409pt}}
\put(635.0,849.0){\rule[-0.200pt]{0.400pt}{2.409pt}}
\put(673.0,131.0){\rule[-0.200pt]{0.400pt}{2.409pt}}
\put(673.0,849.0){\rule[-0.200pt]{0.400pt}{2.409pt}}
\put(691.0,131.0){\rule[-0.200pt]{0.400pt}{4.818pt}}
\put(691,90){\makebox(0,0){ 1000}}
\put(691.0,839.0){\rule[-0.200pt]{0.400pt}{4.818pt}}
\put(747.0,131.0){\rule[-0.200pt]{0.400pt}{2.409pt}}
\put(747.0,849.0){\rule[-0.200pt]{0.400pt}{2.409pt}}
\put(822.0,131.0){\rule[-0.200pt]{0.400pt}{2.409pt}}
\put(822.0,849.0){\rule[-0.200pt]{0.400pt}{2.409pt}}
\put(860.0,131.0){\rule[-0.200pt]{0.400pt}{2.409pt}}
\put(860.0,849.0){\rule[-0.200pt]{0.400pt}{2.409pt}}
\put(878.0,131.0){\rule[-0.200pt]{0.400pt}{4.818pt}}
\put(878,90){\makebox(0,0){ 10000}}
\put(878.0,839.0){\rule[-0.200pt]{0.400pt}{4.818pt}}
\put(934.0,131.0){\rule[-0.200pt]{0.400pt}{2.409pt}}
\put(934.0,849.0){\rule[-0.200pt]{0.400pt}{2.409pt}}
\put(1009.0,131.0){\rule[-0.200pt]{0.400pt}{2.409pt}}
\put(1009.0,849.0){\rule[-0.200pt]{0.400pt}{2.409pt}}
\put(1047.0,131.0){\rule[-0.200pt]{0.400pt}{2.409pt}}
\put(1047.0,849.0){\rule[-0.200pt]{0.400pt}{2.409pt}}
\put(1065.0,131.0){\rule[-0.200pt]{0.400pt}{4.818pt}}
\put(1065,90){\makebox(0,0){ 100000}}
\put(1065.0,839.0){\rule[-0.200pt]{0.400pt}{4.818pt}}
\put(1121.0,131.0){\rule[-0.200pt]{0.400pt}{2.409pt}}
\put(1121.0,849.0){\rule[-0.200pt]{0.400pt}{2.409pt}}
\put(1196.0,131.0){\rule[-0.200pt]{0.400pt}{2.409pt}}
\put(1196.0,849.0){\rule[-0.200pt]{0.400pt}{2.409pt}}
\put(1234.0,131.0){\rule[-0.200pt]{0.400pt}{2.409pt}}
\put(1234.0,849.0){\rule[-0.200pt]{0.400pt}{2.409pt}}
\put(1252.0,131.0){\rule[-0.200pt]{0.400pt}{4.818pt}}
\put(1252,90){\makebox(0,0){ 1e+06}}
\put(1252.0,839.0){\rule[-0.200pt]{0.400pt}{4.818pt}}
\put(1308.0,131.0){\rule[-0.200pt]{0.400pt}{2.409pt}}
\put(1308.0,849.0){\rule[-0.200pt]{0.400pt}{2.409pt}}
\put(1383.0,131.0){\rule[-0.200pt]{0.400pt}{2.409pt}}
\put(1383.0,849.0){\rule[-0.200pt]{0.400pt}{2.409pt}}
\put(1421.0,131.0){\rule[-0.200pt]{0.400pt}{2.409pt}}
\put(1421.0,849.0){\rule[-0.200pt]{0.400pt}{2.409pt}}
\put(1439.0,131.0){\rule[-0.200pt]{0.400pt}{4.818pt}}
\put(1439,90){\makebox(0,0){ 1e+07}}
\put(1439.0,839.0){\rule[-0.200pt]{0.400pt}{4.818pt}}
\put(130.0,131.0){\rule[-0.200pt]{0.400pt}{175.375pt}}
\put(130.0,131.0){\rule[-0.200pt]{315.338pt}{0.400pt}}
\put(1439.0,131.0){\rule[-0.200pt]{0.400pt}{175.375pt}}
\put(130.0,859.0){\rule[-0.200pt]{315.338pt}{0.400pt}}
\put(784,29){\makebox(0,0){buffer size}}
\put(186,131){\usebox{\plotpoint}}
\put(468,130.67){\rule{13.490pt}{0.400pt}}
\multiput(468.00,130.17)(28.000,1.000){2}{\rule{6.745pt}{0.400pt}}
\put(186.0,131.0){\rule[-0.200pt]{67.934pt}{0.400pt}}
\put(693,131.67){\rule{13.490pt}{0.400pt}}
\multiput(693.00,131.17)(28.000,1.000){2}{\rule{6.745pt}{0.400pt}}
\put(749,132.67){\rule{13.731pt}{0.400pt}}
\multiput(749.00,132.17)(28.500,1.000){2}{\rule{6.866pt}{0.400pt}}
\put(806,134.17){\rule{11.300pt}{0.400pt}}
\multiput(806.00,133.17)(32.546,2.000){2}{\rule{5.650pt}{0.400pt}}
\multiput(862.00,136.59)(5.011,0.482){9}{\rule{3.833pt}{0.116pt}}
\multiput(862.00,135.17)(48.044,6.000){2}{\rule{1.917pt}{0.400pt}}
\multiput(918.00,142.59)(3.232,0.489){15}{\rule{2.589pt}{0.118pt}}
\multiput(918.00,141.17)(50.627,9.000){2}{\rule{1.294pt}{0.400pt}}
\multiput(974.00,151.58)(1.103,0.497){49}{\rule{0.977pt}{0.120pt}}
\multiput(974.00,150.17)(54.972,26.000){2}{\rule{0.488pt}{0.400pt}}
\multiput(1031.00,177.58)(0.738,0.498){73}{\rule{0.689pt}{0.120pt}}
\multiput(1031.00,176.17)(54.569,38.000){2}{\rule{0.345pt}{0.400pt}}
\multiput(1087.58,215.00)(0.499,0.688){109}{\rule{0.120pt}{0.650pt}}
\multiput(1086.17,215.00)(56.000,75.651){2}{\rule{0.400pt}{0.325pt}}
\multiput(1143.58,292.00)(0.499,1.505){111}{\rule{0.120pt}{1.300pt}}
\multiput(1142.17,292.00)(57.000,168.302){2}{\rule{0.400pt}{0.650pt}}
\multiput(1200.58,463.00)(0.499,3.087){109}{\rule{0.120pt}{2.557pt}}
\multiput(1199.17,463.00)(56.000,338.693){2}{\rule{0.400pt}{1.279pt}}
\put(186,131){\makebox(0,0){$+$}}
\put(243,131){\makebox(0,0){$+$}}
\put(299,131){\makebox(0,0){$+$}}
\put(355,131){\makebox(0,0){$+$}}
\put(411,131){\makebox(0,0){$+$}}
\put(468,131){\makebox(0,0){$+$}}
\put(524,132){\makebox(0,0){$+$}}
\put(580,132){\makebox(0,0){$+$}}
\put(637,132){\makebox(0,0){$+$}}
\put(693,132){\makebox(0,0){$+$}}
\put(749,133){\makebox(0,0){$+$}}
\put(806,134){\makebox(0,0){$+$}}
\put(862,136){\makebox(0,0){$+$}}
\put(918,142){\makebox(0,0){$+$}}
\put(974,151){\makebox(0,0){$+$}}
\put(1031,177){\makebox(0,0){$+$}}
\put(1087,215){\makebox(0,0){$+$}}
\put(1143,292){\makebox(0,0){$+$}}
\put(1200,463){\makebox(0,0){$+$}}
\put(1256,807){\makebox(0,0){$+$}}
\put(524.0,132.0){\rule[-0.200pt]{40.712pt}{0.400pt}}
\put(186,131){\usebox{\plotpoint}}
\multiput(186,131)(20.756,0.000){3}{\usebox{\plotpoint}}
\multiput(243,131)(20.756,0.000){3}{\usebox{\plotpoint}}
\multiput(299,131)(20.756,0.000){3}{\usebox{\plotpoint}}
\multiput(355,131)(20.756,0.000){2}{\usebox{\plotpoint}}
\multiput(411,131)(20.756,0.000){3}{\usebox{\plotpoint}}
\multiput(468,131)(20.752,0.371){3}{\usebox{\plotpoint}}
\multiput(524,132)(20.756,0.000){2}{\usebox{\plotpoint}}
\multiput(580,132)(20.756,0.000){3}{\usebox{\plotpoint}}
\multiput(637,132)(20.756,0.000){3}{\usebox{\plotpoint}}
\multiput(693,132)(20.752,0.371){3}{\usebox{\plotpoint}}
\multiput(749,133)(20.752,0.364){2}{\usebox{\plotpoint}}
\multiput(806,134)(20.742,0.741){3}{\usebox{\plotpoint}}
\multiput(862,136)(20.637,2.211){3}{\usebox{\plotpoint}}
\multiput(918,142)(20.493,3.293){3}{\usebox{\plotpoint}}
\multiput(974,151)(18.884,8.614){3}{\usebox{\plotpoint}}
\multiput(1031,177)(17.175,11.654){3}{\usebox{\plotpoint}}
\multiput(1087,215)(12.208,16.786){4}{\usebox{\plotpoint}}
\multiput(1143,292)(6.563,19.690){9}{\usebox{\plotpoint}}
\multiput(1200,463)(3.335,20.486){17}{\usebox{\plotpoint}}
\put(1256,807){\usebox{\plotpoint}}
\put(186,131){\makebox(0,0){$\times$}}
\put(243,131){\makebox(0,0){$\times$}}
\put(299,131){\makebox(0,0){$\times$}}
\put(355,131){\makebox(0,0){$\times$}}
\put(411,131){\makebox(0,0){$\times$}}
\put(468,131){\makebox(0,0){$\times$}}
\put(524,132){\makebox(0,0){$\times$}}
\put(580,132){\makebox(0,0){$\times$}}
\put(637,132){\makebox(0,0){$\times$}}
\put(693,132){\makebox(0,0){$\times$}}
\put(749,133){\makebox(0,0){$\times$}}
\put(806,134){\makebox(0,0){$\times$}}
\put(862,136){\makebox(0,0){$\times$}}
\put(918,142){\makebox(0,0){$\times$}}
\put(974,151){\makebox(0,0){$\times$}}
\put(1031,177){\makebox(0,0){$\times$}}
\put(1087,215){\makebox(0,0){$\times$}}
\put(1143,292){\makebox(0,0){$\times$}}
\put(1200,463){\makebox(0,0){$\times$}}
\put(1256,807){\makebox(0,0){$\times$}}
\put(130.0,131.0){\rule[-0.200pt]{0.400pt}{175.375pt}}
\put(130.0,131.0){\rule[-0.200pt]{315.338pt}{0.400pt}}
\put(1439.0,131.0){\rule[-0.200pt]{0.400pt}{175.375pt}}
\put(130.0,859.0){\rule[-0.200pt]{315.338pt}{0.400pt}}
\end{picture}

%    \caption{number of words vs. runtimes. }
    \label{runtimes}
  \end{figure}


\begin{table}[p]
\centering
    \begin{tabular}{ |l|l| }
\hline
        Date                           & Description                                                                  \\ \hline
        Tue Feb 7 17:51:10 2012 -0800  &   added timing code, started writeup           \\
        Tue Feb 7 14:00:36 2012 -0800 &     vector resizes                      \\
        Mon Feb 6 23:59:49 2012 -0800  &     refactor into smaller methods             \\
        Mon Feb 6 23:26:29 2012 -0800 &      Merge branch 'master' of github.com:adamsro/cs311-assign2-uniqify \\
        Mon Feb 6 23:23:42 2012 -0800  &     added transform tolower for in string, added sort and supress functionality        \\
        Sun Feb 5 17:02:37 2012 -0800   &   merge refactor to master \\
        Sun Feb 5 16:09:32 2012 -0800   &   better exception handling. pipes not flushing correctly?        \\
        Sun Feb 5 13:44:17 2012 -0800   &       refactored into class. sorts correctly, no suppressor       \\
        Sat Feb 4 15:21:54 2012 -0800   &    fgets and fputs piped correctly. Correct output for suppressor. infinat fputs loop \\
        Sat Feb 4 09:33:57 2012 -0800   &   working fdopen and fputs? broken fgets?     \\
        Thu Feb 2 17:30:28 2012 -0800   &  vecotrs store pids and pipenums. need fdopen and suppress function.  \\
        Wed Feb 1 23:19:15 2012 -0800   &   general structure. rough, does not compile.       \\
        Fri Jan 27 18:03:51 2012 -0800  &       basic code for forking process      \\
        Fri Jan 27 10:25:10 2012 -0800  &   Hello World!  \\
\hline
    \end{tabular}
\caption{Pulled from git commit log.}\label{commit-logs}
\end{table}


\end{document}
