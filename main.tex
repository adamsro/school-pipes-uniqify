% Robert Adams	02/07/2012	CS 311

\documentclass[letterpaper,10pt]{article} %twocolumn titlepage 
\usepackage{graphicx}
\usepackage{amssymb}
\usepackage{amsmath}
\usepackage{amsthm}

\usepackage{alltt}
\usepackage{float}
\usepackage{color}
\usepackage{url}

\usepackage{balance}
\usepackage[TABBOTCAP, tight]{subfigure}
\usepackage{enumitem}
\usepackage{pstricks, pst-node}


\usepackage{geometry}
\geometry{margin=1in, textheight=8.5in} %textwidth=6in

%random comment

\newcommand{\cred}[1]{{\color{red}#1}}
\newcommand{\cblue}[1]{{\color{blue}#1}}

\usepackage{hyperref}

\def\name{Robert M Adams}

%% The following metadata will show up in the PDF properties
\hypersetup{
  colorlinks = true,
  urlcolor = black,
  pdfauthor = {\name},
  pdfkeywords = {cs311 ``operating systems'' files filesystem I/O},
  pdftitle = {CS 311 Project 2: Sorting Pipes},
  pdfsubject = {CS 311 Project 1},
  pdfpagemode = UseNone
}

\begin{document}
  \title{CS311 Project 2: Sorting Pipes}
  \author{Robert M. Adams}
  \maketitle


\section{Design Decisions}


    To try to write code which was possibly, slightly, more modular I wrote all the primary functionality into a class. Aside from looking cleaner, I am now able to throw Exceptions and handle the formatting all in the same place. The process really rather procedural by nature though, so the run method really only wraps a series of functions which must execute in order.


    I used a struct to hold the child’s id and all pipes then stored a copy of this struct in a vector for every pipe created. This made it much easier for me to keep things organized.  I also created a typedef to group the read and write file descriptors for a process, which I felt made things a bit more clear. I ended up, however, using malloc for the file descriptor array because I wasn’t able to get this working with a vector... probably because I couldn't figure out the syntax.


    As I pulled data from the sort processes pipes I stored the resulting words in a vector. I then iterated through the vector with std::sort, which with the GNU compiler is apparently introsort, about n log n complexity. I then used std::vector<T>.erase with std::unique to remove all duplicates, probably a bit more then ‘n’ time. All and all the process of calling sort multiple times, then loading into an array and calling sort again seems very inefficient, I wasn’t, however, able to come up with a faster way.

\section{Difficulties}


    Keeping track of which pipes go where and when to close what was definitely confusing. If I were to do this again I would try fifos. On a related matter I still haven't figured out why some of my processes are not terminating.  My loop eventuality calls break; and unleashes some zombies.


    When timing the results I found that I can call up to 60 sort processes on smaller files, and the timings are about the same as 1 sort process. I cannot call any more then one sort process on larger files. Although only speculation, I believe this have something  to do with calling sort on the vector. 

  \begin{figure}[p]
    \centering
    % GNUPLOT: LaTeX picture
\setlength{\unitlength}{0.240900pt}
\ifx\plotpoint\undefined\newsavebox{\plotpoint}\fi
\sbox{\plotpoint}{\rule[-0.200pt]{0.400pt}{0.400pt}}%
\begin{picture}(1500,900)(0,0)
\sbox{\plotpoint}{\rule[-0.200pt]{0.400pt}{0.400pt}}%
\put(171.0,131.0){\rule[-0.200pt]{4.818pt}{0.400pt}}
\put(151,131){\makebox(0,0)[r]{ 0}}
\put(1419.0,131.0){\rule[-0.200pt]{4.818pt}{0.400pt}}
\put(171.0,222.0){\rule[-0.200pt]{4.818pt}{0.400pt}}
\put(151,222){\makebox(0,0)[r]{ 0.5}}
\put(1419.0,222.0){\rule[-0.200pt]{4.818pt}{0.400pt}}
\put(171.0,313.0){\rule[-0.200pt]{4.818pt}{0.400pt}}
\put(151,313){\makebox(0,0)[r]{ 1}}
\put(1419.0,313.0){\rule[-0.200pt]{4.818pt}{0.400pt}}
\put(171.0,404.0){\rule[-0.200pt]{4.818pt}{0.400pt}}
\put(151,404){\makebox(0,0)[r]{ 1.5}}
\put(1419.0,404.0){\rule[-0.200pt]{4.818pt}{0.400pt}}
\put(171.0,495.0){\rule[-0.200pt]{4.818pt}{0.400pt}}
\put(151,495){\makebox(0,0)[r]{ 2}}
\put(1419.0,495.0){\rule[-0.200pt]{4.818pt}{0.400pt}}
\put(171.0,586.0){\rule[-0.200pt]{4.818pt}{0.400pt}}
\put(151,586){\makebox(0,0)[r]{ 2.5}}
\put(1419.0,586.0){\rule[-0.200pt]{4.818pt}{0.400pt}}
\put(171.0,677.0){\rule[-0.200pt]{4.818pt}{0.400pt}}
\put(151,677){\makebox(0,0)[r]{ 3}}
\put(1419.0,677.0){\rule[-0.200pt]{4.818pt}{0.400pt}}
\put(171.0,768.0){\rule[-0.200pt]{4.818pt}{0.400pt}}
\put(151,768){\makebox(0,0)[r]{ 3.5}}
\put(1419.0,768.0){\rule[-0.200pt]{4.818pt}{0.400pt}}
\put(171.0,859.0){\rule[-0.200pt]{4.818pt}{0.400pt}}
\put(151,859){\makebox(0,0)[r]{ 4}}
\put(1419.0,859.0){\rule[-0.200pt]{4.818pt}{0.400pt}}
\put(171.0,131.0){\rule[-0.200pt]{0.400pt}{4.818pt}}
\put(171,90){\makebox(0,0){ 1}}
\put(171.0,839.0){\rule[-0.200pt]{0.400pt}{4.818pt}}
\put(226.0,131.0){\rule[-0.200pt]{0.400pt}{2.409pt}}
\put(226.0,849.0){\rule[-0.200pt]{0.400pt}{2.409pt}}
\put(298.0,131.0){\rule[-0.200pt]{0.400pt}{2.409pt}}
\put(298.0,849.0){\rule[-0.200pt]{0.400pt}{2.409pt}}
\put(335.0,131.0){\rule[-0.200pt]{0.400pt}{2.409pt}}
\put(335.0,849.0){\rule[-0.200pt]{0.400pt}{2.409pt}}
\put(352.0,131.0){\rule[-0.200pt]{0.400pt}{4.818pt}}
\put(352,90){\makebox(0,0){ 10}}
\put(352.0,839.0){\rule[-0.200pt]{0.400pt}{4.818pt}}
\put(407.0,131.0){\rule[-0.200pt]{0.400pt}{2.409pt}}
\put(407.0,849.0){\rule[-0.200pt]{0.400pt}{2.409pt}}
\put(479.0,131.0){\rule[-0.200pt]{0.400pt}{2.409pt}}
\put(479.0,849.0){\rule[-0.200pt]{0.400pt}{2.409pt}}
\put(516.0,131.0){\rule[-0.200pt]{0.400pt}{2.409pt}}
\put(516.0,849.0){\rule[-0.200pt]{0.400pt}{2.409pt}}
\put(533.0,131.0){\rule[-0.200pt]{0.400pt}{4.818pt}}
\put(533,90){\makebox(0,0){ 100}}
\put(533.0,839.0){\rule[-0.200pt]{0.400pt}{4.818pt}}
\put(588.0,131.0){\rule[-0.200pt]{0.400pt}{2.409pt}}
\put(588.0,849.0){\rule[-0.200pt]{0.400pt}{2.409pt}}
\put(660.0,131.0){\rule[-0.200pt]{0.400pt}{2.409pt}}
\put(660.0,849.0){\rule[-0.200pt]{0.400pt}{2.409pt}}
\put(697.0,131.0){\rule[-0.200pt]{0.400pt}{2.409pt}}
\put(697.0,849.0){\rule[-0.200pt]{0.400pt}{2.409pt}}
\put(714.0,131.0){\rule[-0.200pt]{0.400pt}{4.818pt}}
\put(714,90){\makebox(0,0){ 1000}}
\put(714.0,839.0){\rule[-0.200pt]{0.400pt}{4.818pt}}
\put(769.0,131.0){\rule[-0.200pt]{0.400pt}{2.409pt}}
\put(769.0,849.0){\rule[-0.200pt]{0.400pt}{2.409pt}}
\put(841.0,131.0){\rule[-0.200pt]{0.400pt}{2.409pt}}
\put(841.0,849.0){\rule[-0.200pt]{0.400pt}{2.409pt}}
\put(878.0,131.0){\rule[-0.200pt]{0.400pt}{2.409pt}}
\put(878.0,849.0){\rule[-0.200pt]{0.400pt}{2.409pt}}
\put(896.0,131.0){\rule[-0.200pt]{0.400pt}{4.818pt}}
\put(896,90){\makebox(0,0){ 10000}}
\put(896.0,839.0){\rule[-0.200pt]{0.400pt}{4.818pt}}
\put(950.0,131.0){\rule[-0.200pt]{0.400pt}{2.409pt}}
\put(950.0,849.0){\rule[-0.200pt]{0.400pt}{2.409pt}}
\put(1022.0,131.0){\rule[-0.200pt]{0.400pt}{2.409pt}}
\put(1022.0,849.0){\rule[-0.200pt]{0.400pt}{2.409pt}}
\put(1059.0,131.0){\rule[-0.200pt]{0.400pt}{2.409pt}}
\put(1059.0,849.0){\rule[-0.200pt]{0.400pt}{2.409pt}}
\put(1077.0,131.0){\rule[-0.200pt]{0.400pt}{4.818pt}}
\put(1077,90){\makebox(0,0){ 100000}}
\put(1077.0,839.0){\rule[-0.200pt]{0.400pt}{4.818pt}}
\put(1131.0,131.0){\rule[-0.200pt]{0.400pt}{2.409pt}}
\put(1131.0,849.0){\rule[-0.200pt]{0.400pt}{2.409pt}}
\put(1203.0,131.0){\rule[-0.200pt]{0.400pt}{2.409pt}}
\put(1203.0,849.0){\rule[-0.200pt]{0.400pt}{2.409pt}}
\put(1240.0,131.0){\rule[-0.200pt]{0.400pt}{2.409pt}}
\put(1240.0,849.0){\rule[-0.200pt]{0.400pt}{2.409pt}}
\put(1258.0,131.0){\rule[-0.200pt]{0.400pt}{4.818pt}}
\put(1258,90){\makebox(0,0){ 1e+06}}
\put(1258.0,839.0){\rule[-0.200pt]{0.400pt}{4.818pt}}
\put(1312.0,131.0){\rule[-0.200pt]{0.400pt}{2.409pt}}
\put(1312.0,849.0){\rule[-0.200pt]{0.400pt}{2.409pt}}
\put(1384.0,131.0){\rule[-0.200pt]{0.400pt}{2.409pt}}
\put(1384.0,849.0){\rule[-0.200pt]{0.400pt}{2.409pt}}
\put(1421.0,131.0){\rule[-0.200pt]{0.400pt}{2.409pt}}
\put(1421.0,849.0){\rule[-0.200pt]{0.400pt}{2.409pt}}
\put(1439.0,131.0){\rule[-0.200pt]{0.400pt}{4.818pt}}
\put(1439,90){\makebox(0,0){ 1e+07}}
\put(1439.0,839.0){\rule[-0.200pt]{0.400pt}{4.818pt}}
\put(171.0,131.0){\rule[-0.200pt]{0.400pt}{175.375pt}}
\put(171.0,131.0){\rule[-0.200pt]{305.461pt}{0.400pt}}
\put(1439.0,131.0){\rule[-0.200pt]{0.400pt}{175.375pt}}
\put(171.0,859.0){\rule[-0.200pt]{305.461pt}{0.400pt}}
\put(30,495){\makebox(0,0){seconds}}
\put(805,29){\makebox(0,0){number of words}}
\put(226,131){\usebox{\plotpoint}}
\put(498,130.67){\rule{13.250pt}{0.400pt}}
\multiput(498.00,130.17)(27.500,1.000){2}{\rule{6.625pt}{0.400pt}}
\put(226.0,131.0){\rule[-0.200pt]{65.525pt}{0.400pt}}
\put(716,131.67){\rule{13.250pt}{0.400pt}}
\multiput(716.00,131.17)(27.500,1.000){2}{\rule{6.625pt}{0.400pt}}
\put(771,132.67){\rule{13.009pt}{0.400pt}}
\multiput(771.00,132.17)(27.000,1.000){2}{\rule{6.504pt}{0.400pt}}
\put(825,134.17){\rule{11.100pt}{0.400pt}}
\multiput(825.00,133.17)(31.961,2.000){2}{\rule{5.550pt}{0.400pt}}
\multiput(880.00,136.59)(4.830,0.482){9}{\rule{3.700pt}{0.116pt}}
\multiput(880.00,135.17)(46.320,6.000){2}{\rule{1.850pt}{0.400pt}}
\multiput(934.00,142.59)(3.174,0.489){15}{\rule{2.544pt}{0.118pt}}
\multiput(934.00,141.17)(49.719,9.000){2}{\rule{1.272pt}{0.400pt}}
\multiput(989.00,151.58)(1.045,0.497){49}{\rule{0.931pt}{0.120pt}}
\multiput(989.00,150.17)(52.068,26.000){2}{\rule{0.465pt}{0.400pt}}
\multiput(1043.00,177.58)(0.725,0.498){73}{\rule{0.679pt}{0.120pt}}
\multiput(1043.00,176.17)(53.591,38.000){2}{\rule{0.339pt}{0.400pt}}
\multiput(1098.58,215.00)(0.499,0.701){107}{\rule{0.120pt}{0.660pt}}
\multiput(1097.17,215.00)(55.000,75.630){2}{\rule{0.400pt}{0.330pt}}
\multiput(1153.58,292.00)(0.498,1.590){105}{\rule{0.120pt}{1.367pt}}
\multiput(1152.17,292.00)(54.000,168.163){2}{\rule{0.400pt}{0.683pt}}
\multiput(1207.58,463.00)(0.499,3.143){107}{\rule{0.120pt}{2.602pt}}
\multiput(1206.17,463.00)(55.000,338.600){2}{\rule{0.400pt}{1.301pt}}
\put(226,131){\makebox(0,0){$+$}}
\put(280,131){\makebox(0,0){$+$}}
\put(335,131){\makebox(0,0){$+$}}
\put(389,131){\makebox(0,0){$+$}}
\put(444,131){\makebox(0,0){$+$}}
\put(498,131){\makebox(0,0){$+$}}
\put(553,132){\makebox(0,0){$+$}}
\put(607,132){\makebox(0,0){$+$}}
\put(662,132){\makebox(0,0){$+$}}
\put(716,132){\makebox(0,0){$+$}}
\put(771,133){\makebox(0,0){$+$}}
\put(825,134){\makebox(0,0){$+$}}
\put(880,136){\makebox(0,0){$+$}}
\put(934,142){\makebox(0,0){$+$}}
\put(989,151){\makebox(0,0){$+$}}
\put(1043,177){\makebox(0,0){$+$}}
\put(1098,215){\makebox(0,0){$+$}}
\put(1153,292){\makebox(0,0){$+$}}
\put(1207,463){\makebox(0,0){$+$}}
\put(1262,807){\makebox(0,0){$+$}}
\put(553.0,132.0){\rule[-0.200pt]{39.267pt}{0.400pt}}
\put(226,131){\usebox{\plotpoint}}
\multiput(226,131)(20.756,0.000){3}{\usebox{\plotpoint}}
\multiput(280,131)(20.756,0.000){3}{\usebox{\plotpoint}}
\multiput(335,131)(20.756,0.000){2}{\usebox{\plotpoint}}
\multiput(389,131)(20.756,0.000){3}{\usebox{\plotpoint}}
\multiput(444,131)(20.756,0.000){3}{\usebox{\plotpoint}}
\multiput(498,131)(20.752,0.377){2}{\usebox{\plotpoint}}
\multiput(553,132)(20.756,0.000){3}{\usebox{\plotpoint}}
\multiput(607,132)(20.756,0.000){3}{\usebox{\plotpoint}}
\multiput(662,132)(20.756,0.000){2}{\usebox{\plotpoint}}
\multiput(716,132)(20.752,0.377){3}{\usebox{\plotpoint}}
\multiput(771,133)(20.752,0.384){2}{\usebox{\plotpoint}}
\multiput(825,134)(20.742,0.754){3}{\usebox{\plotpoint}}
\multiput(880,136)(20.629,2.292){3}{\usebox{\plotpoint}}
\multiput(934,142)(20.483,3.352){2}{\usebox{\plotpoint}}
\multiput(989,151)(18.701,9.004){3}{\usebox{\plotpoint}}
\multiput(1043,177)(17.076,11.798){3}{\usebox{\plotpoint}}
\multiput(1098,215)(12.064,16.889){5}{\usebox{\plotpoint}}
\multiput(1153,292)(6.250,19.792){9}{\usebox{\plotpoint}}
\multiput(1207,463)(3.277,20.495){16}{\usebox{\plotpoint}}
\put(1262,807){\usebox{\plotpoint}}
\put(226,131){\makebox(0,0){$\times$}}
\put(280,131){\makebox(0,0){$\times$}}
\put(335,131){\makebox(0,0){$\times$}}
\put(389,131){\makebox(0,0){$\times$}}
\put(444,131){\makebox(0,0){$\times$}}
\put(498,131){\makebox(0,0){$\times$}}
\put(553,132){\makebox(0,0){$\times$}}
\put(607,132){\makebox(0,0){$\times$}}
\put(662,132){\makebox(0,0){$\times$}}
\put(716,132){\makebox(0,0){$\times$}}
\put(771,133){\makebox(0,0){$\times$}}
\put(825,134){\makebox(0,0){$\times$}}
\put(880,136){\makebox(0,0){$\times$}}
\put(934,142){\makebox(0,0){$\times$}}
\put(989,151){\makebox(0,0){$\times$}}
\put(1043,177){\makebox(0,0){$\times$}}
\put(1098,215){\makebox(0,0){$\times$}}
\put(1153,292){\makebox(0,0){$\times$}}
\put(1207,463){\makebox(0,0){$\times$}}
\put(1262,807){\makebox(0,0){$\times$}}
\put(171.0,131.0){\rule[-0.200pt]{0.400pt}{175.375pt}}
\put(171.0,131.0){\rule[-0.200pt]{305.461pt}{0.400pt}}
\put(1439.0,131.0){\rule[-0.200pt]{0.400pt}{175.375pt}}
\put(171.0,859.0){\rule[-0.200pt]{305.461pt}{0.400pt}}
\end{picture}

%    \caption{number of words vs. runtimes. }
    \label{runtimes}
  \end{figure}


\begin{table}[p]
\centering
    \begin{tabular}{ |l|l| }
\hline
        Date                           & Description                                                                  \\ \hline
        Tue Feb 7 17:51:10 2012 -0800  &   added timing code, started writeup           \\
        Tue Feb 7 14:00:36 2012 -0800 &     vector resizes                      \\
        Mon Feb 6 23:59:49 2012 -0800  &     refactor into smaller methods             \\
        Mon Feb 6 23:26:29 2012 -0800 &      Merge branch 'master' of github.com:adamsro/cs311-assign2-uniqify \\
        Mon Feb 6 23:23:42 2012 -0800  &     added transform tolower for in string, added sort and supress functionality        \\
        Sun Feb 5 17:02:37 2012 -0800   &   merge refactor to master \\
        Sun Feb 5 16:09:32 2012 -0800   &   better exception handling. pipes not flushing correctly?        \\
        Sun Feb 5 13:44:17 2012 -0800   &       refactored into class. sorts correctly, no suppressor       \\
        Sat Feb 4 15:21:54 2012 -0800   &    fgets and fputs piped correctly. Correct output for suppressor. infinat fputs loop \\
        Sat Feb 4 09:33:57 2012 -0800   &   working fdopen and fputs? broken fgets?     \\
        Thu Feb 2 17:30:28 2012 -0800   &  vecotrs store pids and pipenums. need fdopen and suppress function.  \\
        Wed Feb 1 23:19:15 2012 -0800   &   general structure. rough, does not compile.       \\
        Fri Jan 27 18:03:51 2012 -0800  &       basic code for forking process      \\
        Fri Jan 27 10:25:10 2012 -0800  &   Hello World!  \\
\hline
    \end{tabular}
\caption{Pulled from git commit log.}\label{commit-logs}
\end{table}


\end{document}
